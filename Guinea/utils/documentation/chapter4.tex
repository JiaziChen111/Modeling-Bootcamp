\section{Préparation des données}
\begin{enumerate}
    {\bf \item  Mise à jour et amélioration des bases de données brutes}. 
    \onehalfspacing Au stade actuel, nos bases de données se composent de séries chronologiques mensuelles, trimestrielles et annuelles conservées au format * .csv.
     Les noms de fichiers suivent: {\color{magenta}db\_monthly.csv}, {\color{magenta}db\_quarterly.csv}, {\color{magenta}db\_yearly.csv}. Les fichiers doivent être mis à jour à chaque tour de prévision. 
%    There is one important thing about series: all seasonally unadjusted series must have '\_su' suffix in the end, if you want them to be adjusted automatically. 
Assurez-vous d'utiliser des points comme des décimales (par exemple, 9.5 au lieu de 9,5).
    {\bf \item Lecture et traitement des données}. 
    Le traitement des données est effectué dans le fichier {\color{magenta} a1\_model\_data.m}. Les actions clés du fichier suivent:
    \begin{itemize}
        \item lire les fichiers de bases de données brutes ({\color{magenta}db\_monthly.csv} etc.) du dossier {\color{blue} donnes}
        \item ajustement saisonnier des données mensuelles et trimestrielles
        \item convertir les données de la base de données mensuelle et annuel en données trimestrielles
        \item transformer les données en logs (multipliés par 100), en croissance annuelle si néccessaire
%        \item providing special adjustment of indicators (e.g. for tobacco prices using the ad-hoc model)
        \item préparer des indicateurs spéciaux qui seront utilisés dans le MPT en tant qu'observables (termes de l'échange, PIB étranger effectif, IPC étranger effectif, etc.) 
%        \item creating dictionaries for name of the variable and its description
        \item enregistrer la base de données des variables observées pour le modèle
    \end{itemize}
%    Most of seasonal adjusment, conversion to logs and taking differences are executed automatically (in loops). 
%    For instance, seasonal adjustment are done in a loop for indicators that have '\_su'
%    in the end of their name (the loop starts from checking if variable has this feature).\\
    {\bf Entrées et sorties.}
    \begin{itemize}
    \item[\textrightarrow] {\color{magenta}donnees\textbackslash db\_monthly.csv},
    {\color{magenta}donnees\textbackslash db\_quarterly.csv},
    {\color{magenta}donnees\textbackslash db\_yearly.csv},
    {\color{magenta}donnees\textbackslash db\_external.csv}
    \item[\textleftarrow] {\color{magenta}results\textbackslash a1\_model\_data \textbackslash observed\_db.mat}
    \end{itemize}
    %{\bf Auxiliary files}
%    \begin{itemize}
%        \item[\textrightarrow] {\color{magenta}utils\textbackslash read\_tobacco.m}
%        \item[\textrightarrow] {\color{magenta}models\textbackslash read\_tobacco.model}
%        \item[\textrightarrow] {\color{magenta}utils\textbackslash barstacked.m}
%    \end{itemize}
\end{enumerate}